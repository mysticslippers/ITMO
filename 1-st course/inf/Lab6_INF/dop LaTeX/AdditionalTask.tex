\documentclass[12pt,onecolumn]{article}

\usepackage[english, russian]{babel}
\usepackage[T2A]{fontenc}
\usepackage[utf8]{inputenc}
\usepackage{graphicx}
\usepackage[unicode, pdftex]{hyperref}
\usepackage{amssymb}
\usepackage{blindtext}
\usepackage{float}
\usepackage{multicol}
\usepackage{import}
\usepackage{xifthen}
\usepackage{pdfpages}
\usepackage{transparent}
\usepackage{tikz}
\usepackage{xcolor}
\title{Дополнительное задание по \LaTeX.}
\author{Михайлов Дмитрий Андреевич, P3118}

\begin{document}
	\maketitle
	
	\begin{multicols}{2}
		\begin{tikzpicture}[xscale=2.0, yscale=0.8]
			\draw[very thick, left color=blue, right color=red] (0, 0) -- (3, 0) -- (3, 4) -- cycle;
		\end{tikzpicture}
		\linebreak
	
		Египетский треугольник — прямоугольный треугольник с соотношением сторон 3:4:5.
	\end{multicols}
	\vspace*{10mm}
	
	\begin{multicols}{2}
		\begin{tikzpicture}[xscale=1.0, yscale=1.0]
			\draw[very thick, left color=green, right color=yellow] (0, 0) circle (1.5);
		\end{tikzpicture}
		\linebreak
		
		Окружность — замкнутая плоская кривая, которая состоит из всех точек на плоскости, равноудалённых от заданной точки, лежащей в той же плоскости, что и кривая: эта точка называется центром окружности.
	\end{multicols}
	\vspace*{10mm}
	
	\begin{multicols}{2}
		\begin{tikzpicture}[xscale=1.5, yscale=1.0]
			\draw[very thick, left color=red, right color=orange] (0, 0) rectangle (2, 2);
		\end{tikzpicture}
		\linebreak
		Квадрат — правильный четырёхугольник, то есть плоский четырёхугольник, у которого все углы и все стороны равны.
	\end{multicols}
\thispagestyle{empty}
\end{document}